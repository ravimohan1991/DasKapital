\documentclass[12pt]{extarticle}
\usepackage{hyperref}
\usepackage{amsmath}
\usepackage{amsfonts}
\usepackage[dvips]{graphicx}
\usepackage{mdframed}
\usepackage[retainorgcmds]{IEEEtrantools}
\usepackage{amssymb}
\usepackage{amsthm}
\usepackage[utf8]{inputenc}
\usepackage[english]{babel}
\usepackage{color}
\usepackage{verbatim}
\usepackage{media9}
\usepackage{xcolor}

 
\newtheorem{theorem}{Theorem}[section]
\newtheorem{corollary}{Corollary}[theorem]
\newtheorem{lemma}[theorem]{Lemma}
\theoremstyle{definition}
\newtheorem{definition}{Definition}[section]
\newenvironment{remark}[1][Remark]{\begin{trivlist}
\item[\hskip \labelsep {\bfseries #1}]}{\end{trivlist}}
 
\begin{document}


\section{Definitions}
This section consists of abstractions which will be summoned in the rest of the material.  Please try to be as cooperative as possible with them.

\begin{definition}
  A \emph{Society} is a form? consisting of collective \emph{conciousness} of \emph{entities} or \emph{individuals} along with the mecanisms of sustenance for \emph{human} life form.  
  \end{definition}

\begin{definition}
  In \emph{Capitalist} mode of production in a \emph{Society}, the wealth is immense accumulation of commodities.
\end{definition}

\begin{definition}
  \label{def:commodity}
  A \emph{commodity} is an abstraction represented by an \emph{object} such that it
  \begin{itemize}
  \item is outside \emph{human} context
  \item results in human satisfaction
  \item represents collection of \emph{some}\footnote{The nature of some will be revealed elsewhere within these notes.} properties
  \end{itemize} 
\end{definition}

\begin{definition}
  A \emph{utility} of a commodity is the resultant of physical properties of that commodity.
\end{definition}

\begin{definition}
 \label{def:thing}
  A \emph{thing} is a physical\footnote{Meaning the context obeying ``Laws of Physics''.} product.  Examples are silicon chip, figurine, and energy drinks.
\end{definition}
Note: every useful thing has two point of views: \emph{qualitative} and \emph{quantitative}.

\begin{definition}
  \label{def:usevalue}
  The utility of a thing\footnote{The ``thing'' could be ``product of human labor''} makes the thing a \emph{use value}.
  \end{definition}

  \begin{lemma}
    Commodity is a use value.
    \end{lemma}
  \begin{proof}
    From definitions \ref{def:thing} and \ref{def:usevalue} it is easy to see that once we use the qualitative(?) viewpoint, commodity is essentially a use value.
  \end{proof}
  \begin{remark}
    We note that use value is now a property of the commodity defined in definition \ref{def:commodity}.  Further more we demand/observe that the use value
    \begin{itemize}
    \item (\textcolor{blue}{Schezophrenia Alert!!!}) is \textbf{completely} independent of the amount of \textcolor{red}{labor}(needs clarification on the definition here) spent into the commodity to appropriate its properties.  Not really sure if Karl is referring to use value or the virtue that utility of commodity manifests itself as use value that is independent of the labor.
    \item  is associated with \emph{definite} quantities for instance couple of transistors, six-pack of monster drink, and dozen of static meshes.
    \item is for gauging the commercial knowledge associated with the commodity.
    \item becomes reality \emph{only} on consumption.
    \item constitute the substance of all wealth (irrespective of the social form of the wealth).
    \item A thing can be a use value to its producer, and product of human labor, without being a commodity.
      \item In order for that product to become commodity, it must have use value for \emph{others}.  Note that this is in concurrence with the definition \ref{def:commodity}.
    \end{itemize}
  \end{remark}

  \begin{definition}
    \label{def:exchval}
    \emph{Exchange value} is a quantitative relation which relates the value of one sort to that of other.\footnote{Here we have defined exchange value in very generic form.  No reference to previous terminologies is supposed to be invoked.}  The relation constantly changes with \emph{time} and \emph{place}.
  \end{definition}

  \begin{remark}
    We note that
    \begin{itemize}
    \item Exchange value \emph{appears} to be something accidental.
    \item Purely \emph{relative}.
    \item One sort can have different exchange values depending on what other sorts it is being compared with.
    \item A \emph{natural} step would be to assume exchange values be transitive.
      \item I suspect that the existence of multiple commodities in the same context gives rise to notion of exchange vlaues.
    \end{itemize}
    \end{remark}

    \begin{lemma}
      Exchange value is intrinsic to the commodities.\footnote{It seems in direct contradiction with the statement ``Nothing can have intrinsik value'' \cite{Barbon:777}.}
    \end{lemma}

    \begin{proof}
      It is quite easy to see that.
      \end{proof}

      \begin{remark}
        We note that
        \begin{itemize}
        \item Exchange values arise in the quantitative viewpoint of commodities.
          \item Exchange values are completely independent from use values.
          \end{itemize}
        \end{remark}

        \begin{definition}
          \emph{Human labor} is an abstract property of commodity which is devoid of any qualitative substance\footnote{Getting rid of use values of the product in this context.  Or is he?  Later he writes that ``Values are abstracted from use values''.  Is this acceptable?} and useful character of any definite kind of labor (like that of mason, porter, and builder) along with their concerete form, and which is common to all the commodities viewed as \emph{product of labor}.
        \end{definition}

        \begin{remark}
        We note that
        \begin{itemize}
        \item the definition is independent of whatever change of hands the commodity goes through.
        \item the defintion is concerned with the commodity as product of labor by virtue of the commodity's essence and not any qualitative viewpoint.
        \item human labor is \emph{homogeneous} with respect to the commodity.
        \end{itemize}
      \end{remark}

      \begin{definition}
        \emph{Values} are mere congelation (semi-solidification) of homogeneous human labor and \emph{labor-power}\footnote{The labor-power is the \emph{reservior} of the human labor.} expended irrespective of \emph{mode} of its expenditure.
      \end{definition}

      \begin{remark}
        We note that
        \begin{itemize}
        \item Karl fished out this definition of Values by some ``Social Substance crystal''.  I need to understand what it exactly is.
          \item At this point he kind of stresses that Values are abstracted from use values?
        \end{itemize}
      \end{remark}

      \begin{lemma}
        Values of commodities are expressed/manifest themselves in form of exchange values when commodities are exchanged.
      \end{lemma}

      \begin{proof}
        From definition \ref{def:exchval} (of exchange value) we demand that they be transitive.  As the consequence, when commodities (with of course different exchange values) are exchanged, they would still have a common substance, which we can \emph{naturally} identify as Value. 
      \end{proof}

      \begin{remark}
        We make the folowing statements
        \begin{itemize}
        \item The magnitude of value can be measured by the homogeneous human labor spent on the commodity, which, in turn, is measured by its duration.
        \item The duration can be ``denominated'' by weeks, days, and hours.
          \item Finally, we think and ponder on how to find physical representation of true homogeneous human labor.\footnote{Clearly a procrastinating worker completing the work in indefinite amount of time doesn't enhance the value of commodity.  If anything, it diminishes the value.}  
          \end{itemize} 
        \end{remark}

        \begin{definition}
          The total \emph{labor-power} of a society is the \emph{sum total} of the values of \emph{all} the commodities produced by the same society.  This is the homogenoeus mass of uniform human labor-power.
        \end{definition}

        \begin{remark}
          We note that
          \begin{itemize}
          \item Labor-power is \emph{thought to be} composed of innumerable individual \emph{units}\footnote{\textcolor{red}{The detailed analysis on \emph{unit} should be revealed in the later part of these notes.}}.
            \item Futhermore we assume (in the context of commodity production) that these units are indistinguisahble from each other.  Each representative of \emph{some} average colletive property (of society).
            \item Units instrument the gauging of social standards.  That is, what are the regular working timings and so on.
          \end{itemize}
        \end{remark}
        
        Above discussion begs and then demands the right context for homogeneous human labor representation.  One individual may or maynot be enough, but from experience and Karl's arguments, it \emph{seems}, the right context arises with the \emph{collective contribution} of entities existing in a society.

        Let me explain and elaborate Karl's arguments to generate greater mass appeal!  The substance that generates the value (of commodity) is the uniform expenditure of a unit(?) uniform labor-power in form of homogeneous human labor.  Then given the context of commodity production in a society, the value of a commodity is dictated by the expenditure of human labor-power within the time period\footnote{The period is the one that is \emph{socially necessary} given the socially normal conditions of production with the average degree of skill and intensity prevalant at that time.} resulting from the social standards.  Therefore an individual spending less or more time than that socially gauged magnitude is not really representing the homogenous human labor\footnote{Either he is God or highly irrational person!}.

        \paragraph{Case Study: Spinning Jenny}
        The introduction of power looms in England, around 1765 (\href{https://en.wikipedia.org/wiki/James_Hargreaves}{James Hargreaves}), led to the reduced work for churning yarn into cloth.  Thus, the socially gauged magnitude of time for the commodity production, naturally(?) halved (or some fraction).  On the other hand, the handloom weavers, produced the same commodity in same time as before.  That equates to half of the produce, by social standards, given the same time, thus half the human labor (remember it is homogeneous in all possible ways) expended on handloom materials and cosequently, now, the handloom commodity value is halved.

        \begin{remark}
          We note that
          \begin{itemize}
          \item Since social standards now dictate the time, conditions, and so on for commodity production, we have a context for various possible \emph{social dynamics} to appear and chalk out those standards, which in turn drive the society in \emph{some} direction, in very non-direct sense.
            \item It is easy to see that one of the dynamics can be competition which can and will spark in this context, which can be both blessing and curse\footnote{I take no responsibility for describing competition social dynamics here, because my knowledge is quite incomplete and personally I feel it is highly stupid, in the current form, but existing notion.}.
          \end{itemize}
        \end{remark}

        \begin{lemma}
          As values, all commodities are only definite masses of congealed labor time.
        \end{lemma}
        \begin{proof}
          Easy to visualize from above text and discussions.
        \end{proof}

        Unproved lemma
        \begin{lemma}
          \textcolor{red}{In the context of working standards dictated by social standards, each commodity is to be considered as an average sample of its class.}
        \end{lemma}
        \begin{proof}
          To be chalked out elsewhere, or here elsewhen!
        \end{proof}
        \begin{remark}
          \label{rem:comclass}
          We note that
          \begin{itemize}
          \item I don't know what \emph{class} means here.
          \item This lemma deals with social \emph{viewpoint} of the commodity.
            \item In this context, commodities become indistiguisable \emph{if} produced in the \emph{same amount}\footnote{Karl is not serious about the quality of labor time though, here.  It seems that time is outside to his theory and taken to be granted no matter where it comes from.  I'd say that be dictated by the nature of society.} of time.
          \end{itemize}
        \end{remark}

\section{Origins of Definitions and their contexts}
I won't recommend this section for crash course'ers.  Only if you have enough patience and care, you may proceed.
\subsection{Two-fold nature of labor}
The homogeneous human labor in abstract can be expressed in
\begin{itemize}
\item use value
\item value
\end{itemize}
of a commodity with very different characteristics.  Karl claims that it is due to the fundamental \emph{two-fold} nature of labor itself when studied in the context of commodities.

Consider two commodities, as use values, as following
\begin{itemize}
\item a coat $\Phi$
\item 10 yards of linen $\Xi$
\end{itemize}
and let
\begin{equation}
  \Phi = 2v \text{ and } \Xi = v,
\end{equation}
where, $v$ is some value.

\begin{definition}
  \label{def:usefullabor}
  The labor which is responsible for making a thing (\ref{def:thing}), a use value, is called \emph{useful labor}.
\end{definition}

\begin{remark}
  We note that:
  \begin{itemize}
  \item Useful labor maifests itself in the utility of the thing (which makes it a use value).
    \item Useful labor is productive activity of definite kind and exercised with a definite aim.
  \item In the above example, $\Phi$ and $\Xi$ are different from qualitative viewpoint.  This implies that the labor responsible for them, correspondingly, should be of different forms i.e tailoring and weaving.
  \item There is no point of exchange between same (indistinguisahble) use values\footnote{For the source of indistinguisable context, see \ref{rem:comclass}.}, seen from qualitative viewpoint.
  \item It is natural to associate a bijection (?) with different use values and useful labor.
  \item This demands a calssification of useful labor according to the order, genus, species, and variety to which they belong in the social division of labor.
    \item\label{it:divoflabor}  In community of commodity production, the qualitative difference of useful labor, existing in variety of forms and carried on independently by \emph{individual} producers with accountability, developes into a system primarily supported by a \emph{structure} called division of labor.
  \end{itemize}
\end{remark}


``\emph{\textcolor{blue}{The division of labor is necessary condition for the commodity production.  The commodity production is not necessary condition for divison of labor.}}''\footnote{This makes me realize that division of labor is a structure rather than a mere condition.  See remark \ref{it:divoflabor}.}

For latter, different factors/systems such as biology may be at play\footnote{Think in evolutionary terms and that might lead to something interesting.}.

``\emph{\textcolor{blue}{That (physical product) which is not a spontaneous produce of Nature, shall owe their existence to some special productive activity, with a definite aim, that appropriates the naturally occuring materials to particular human wants.}}''

Now, as long as labor's useful labor characterstic is manifest, it is a necessary and Nature imposed condition (constraint, independent of all forms of society) for the existence of human race, without which there is no material exchange between human and Nature, and, consequently, no life.

``\emph{\textcolor{blue}{The relation between physical product (or use value or commodity?) and labor is invariant of the nature of useful labor, in the sense, that it might be special, by being independent branch of the social division of labor}}.''

\bibliographystyle{JHEP.bst}
\bibliography{list.bib}
\end{document}