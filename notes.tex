\documentclass[12pt]{extarticle}
\usepackage{hyperref}
\usepackage{amsmath}
\usepackage{amsfonts}
\usepackage[dvips]{graphicx}
\usepackage{mdframed}
\usepackage[retainorgcmds]{IEEEtrantools}
\usepackage{amssymb}
\usepackage{amsthm}
\usepackage[utf8]{inputenc}
\usepackage[english]{babel}
\usepackage{color}
\usepackage{verbatim}
\usepackage{media9}

 
\newtheorem{theorem}{Theorem}[section]
\newtheorem{corollary}{Corollary}[theorem]
\newtheorem{lemma}[theorem]{Lemma}
\theoremstyle{definition}
\newtheorem{definition}{Definition}[section]
\newenvironment{remark}[1][Remark]{\begin{trivlist}
\item[\hskip \labelsep {\bfseries #1}]}{\end{trivlist}}
 
\begin{document}


\section{Definitions}
This section consists of abstractions which will be summoned in the rest of the material.  Please try to be as cooperative as possible with them.

\begin{definition}
  A \emph{Society} is a form? consisting of collective \emph{conciousness} of \emph{entities} or \emph{individuals} along with the mecanisms of sustainance for \emph{human} life form.  
  \end{definition}

\begin{definition}
  In \emph{Capitalist} mode of production in a \emph{Society}, the wealth is immense accumulation of commodities.
\end{definition}

\begin{definition}
  \label{def:commodity}
  A \emph{commodity} is an abstraction represented by an \emph{object} such that it
  \begin{itemize}
  \item is outside \emph{human} context
  \item results in human satisfaction
  \item represents collection of \emph{some}\footnote{The nature of some will be revealed elsewhere within these notes.} properties
  \end{itemize} 
\end{definition}

\begin{definition}
  A \emph{utility} of a commodity is the resultant of physical properties of that commodity.
\end{definition}

\begin{definition}
 \label{def:thing}
  A \emph{thing} is certain form of commodity?  Examples are silicon chip, figurine, and energy drinks.
\end{definition}
Note: every useful thing has two point of views: \emph{qualitative} and \emph{quantitative}.

\begin{definition}
  \label{def:usevalue}
  The utility of a thing makes the thing a \emph{use value}.
  \end{definition}

  \begin{lemma}
    Commodity is a use value.
    \end{lemma}
  \begin{proof}
    From definitions \ref{def:thing} and \ref{def:usevalue} it is easy to see that once we use the qualitative(?) viewpoint, commodity is essentially a use value.
  \end{proof}
  \begin{remark}
    We note that use value is now a property of the commodity defined in definition \ref{def:commodity}.  Further more we demand/observe that the use value
    \begin{itemize}
    \item is \textbf{completely} independent of the amount of labor spent into the commodity.
    \item  is associated with \emph{definite} quantities for instance couple of transistors, six-pack of monster drink, and dozen of static meshes.
    \item is for gauging the commercial knowledge associated with the commodity.
    \item becomes reality \emph{only} on consumption.
      \item constitute the substance of all wealth (irrespective of the social form of the wealth).
    \end{itemize}
  \end{remark}

  \begin{definition}
    \emph{Exchange value} is a quantitative relation which relates the value of one sort to that of other.\footnote{Here we have defined exchange value in very generic form.  No reference to previous terminologies is supposed to be invoked.}  The relation constantly changes with \emph{time} and \emph{place}.
  \end{definition}

  \begin{remark}
    We note that
    \begin{itemize}
    \item Exchange value \emph{appears} to be something accidental.
    \item Purely \emph{relative}.
    \item One sort can have different exchange values depending on what other sorts it is being compared with.
    \item A \emph{natural} step would be to assume exchange values be transitive.
      \item I suspect that the existence of multiple commodities in the same context gives rise to notion of exchange vlaues.
    \end{itemize}
    \end{remark}

    \begin{lemma}
      Exchange value is intrinsic to the commodities.\footnote{It seems in direct contradiction with the statement ``Nothing can have intrinsik value'' \cite{Barbon:777}.}
    \end{lemma}

    \begin{proof}
      It is quite easy to see that.
      \end{proof}

      \begin{remark}
        We note that
        \begin{itemize}
        \item Exchange values arise in the quantitative viewpoint of commodities.
          \item Exchange values are completely independent from use values.
          \end{itemize}
        \end{remark}
\bibliographystyle{JHEP.bst}
\bibliography{list.bib}
\end{document}