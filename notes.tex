\documentclass[12pt]{extarticle}
\usepackage{hyperref}
\usepackage{amsmath}
\usepackage{amsfonts}
\usepackage[dvips]{graphicx}
\usepackage{mdframed}
\usepackage[retainorgcmds]{IEEEtrantools}
\usepackage{amssymb}
\usepackage{amsthm}
\usepackage[utf8]{inputenc}
\usepackage[english]{babel}
\usepackage{color}
\usepackage{verbatim}
\usepackage{media9}
\usepackage{xcolor}
\usepackage{titlesec}

% Useful definitions for Math environments 
\newtheorem{theorem}{Theorem}[section]
\newtheorem{corollary}{Corollary}[theorem]
\newtheorem{lemma}[theorem]{Lemma}
\theoremstyle{definition}
\newtheorem{definition}{Definition}[section]
\newenvironment{remark}[1][Remark]{\begin{trivlist}
\item[\hskip \labelsep {\bfseries #1}]}{\end{trivlist}}

% For generating subsubsub sections
\setcounter{secnumdepth}{4}
\titleformat{\paragraph}
{\normalfont\normalsize\bfseries}{\theparagraph}{1em}{}
\titlespacing*{\paragraph}
{0pt}{3.25ex plus 1ex minus .2ex}{1.5ex plus .2ex}

% For replacing red link boxes with sturdy colors
\hypersetup{
    colorlinks,
    linkcolor={red!50!black},
    citecolor={blue!50!black},
    urlcolor={blue!80!black}
}

\begin{document}

\tableofcontents

\section{Definitions}
This section consists of abstractions which will be summoned in the rest of the material.  Please try to be as cooperative as possible with them.

\begin{definition}
  A \emph{Society} is a form? consisting of collective \emph{conciousness} of \emph{entities} or \emph{individuals} along with the mecanisms of sustenance for \emph{human} life form.  
  \end{definition}

\begin{definition}
  In \emph{Capitalist} mode of production in a \emph{Society}, the wealth is immense accumulation of commodities.
\end{definition}

\begin{definition}
  \label{def:commodity}
  A \emph{commodity} is an abstraction represented by an \emph{object} such that it
  \begin{itemize}
  \item is outside \emph{human} context
  \item results in human satisfaction
  \item represents collection of \emph{some}\footnote{The nature of some will be revealed elsewhere within these notes.} properties
  \end{itemize} 
\end{definition}

\begin{remark}
  We note that:
  \begin{itemize}
  \item Commodity has two-fold nature, in the sense that, it is object of utility \ref{def:utility}, at the same time, repository of value \ref{def:values}.
  \end{itemize}
\end{remark}

\begin{definition}
  \label{def:utility}
  A \emph{utility} of a commodity is the resultant of physical properties of that commodity.
\end{definition}

\begin{definition}
 \label{def:thing}
  A \emph{thing} is a physical\footnote{Meaning the context obeying ``Laws of Physics''.} product in its entirety.  Examples are silicon chip, figurine, and energy drink.
\end{definition}
Note: every useful thing has two point of views: \emph{qualitative} and \emph{quantitative}.

\begin{definition}
  \label{def:usevalue}
  The utility of a thing\footnote{The ``thing'' could be ``product of human labor''} makes the thing a \emph{use value}.
  \end{definition}

  \begin{lemma}
    Commodity is a use value.
    \end{lemma}
  \begin{proof}
    From definitions \ref{def:thing} and \ref{def:usevalue} it is easy to see that once we use the qualitative(?) viewpoint, commodity is essentially a use value.
  \end{proof}
  \begin{remark}
    We note that use value is now a property of the commodity defined in definition \ref{def:commodity}.  Further more we demand/observe that the use value
    \begin{itemize}
    \item (\textcolor{blue}{Schezophrenia Alert!!!}) is \textbf{completely} independent of the amount of \textcolor{red}{labor}(needs clarification on the definition here) spent into the commodity to appropriate its properties.  Not really sure if Karl is referring to use value or the virtue that utility of commodity manifests itself as use value that is independent of the labor.
    \item  is associated with \emph{definite} quantities for instance couple of transistors, six-pack of monster drink, and dozen of static meshes.
    \item is for gauging the commercial knowledge associated with the commodity.
    \item becomes reality \emph{only} on consumption.
    \item constitute the substance of all wealth (irrespective of the social form of the wealth).
    \item A thing can be a use value to its producer, and product of human labor, without being a commodity.
      \item In order for that product to become commodity, it must have use value for \emph{others}.  Note that this is in concurrence with the definition \ref{def:commodity}.
    \end{itemize}
  \end{remark}

  \begin{definition}
    \label{def:exchval}
    \emph{Exchange value} is a quantitative relation which relates the value of one sort to that of other.\footnote{Here we have defined exchange value in very generic form.  No reference to previous terminologies is supposed to be invoked.}  The relation constantly changes with \emph{time} and \emph{place}.
  \end{definition}

  \begin{remark}
    We note that
    \begin{itemize}
    \item Exchange value \emph{appears} to be something accidental.
    \item Purely \emph{relative}.
    \item One sort can have different exchange values depending on what other sorts it is being compared with.
    \item A \emph{natural} step would be to assume exchange values be transitive.
      \item I suspect that the existence of multiple commodities in the same context gives rise to notion of exchange vlaues.
    \end{itemize}
    \end{remark}

    \begin{lemma}
      \label{lemma:exval}
      Exchange value is intrinsic to the commodities.\footnote{It seems in direct contradiction with the statement ``Nothing can have intrinsik value'' \cite{Barbon:777}.}
    \end{lemma}

    \begin{proof}
      It is quite easy to see that.
      \end{proof}

      \begin{remark}
        We note that
        \begin{itemize}
        \item Exchange values arise in the quantitative viewpoint of commodities.
          \item Exchange values are completely independent from use values.
          \end{itemize}
        \end{remark}

        \begin{definition}
          \emph{Human labor} is an abstract property of commodity which is devoid of any qualitative substance\footnote{Getting rid of use values of the product in this context.} and useful character of any definite kind of labor (like that of mason, porter, and weaver) along with their concerete form, and which is common to all the commodities viewed as \emph{product of labor}.
        \end{definition}

        \begin{remark}
        We note that
        \begin{itemize}
        \item the definition is independent of whatever change of hands the commodity goes through.
        \item the defintion is concerned with the commodity as product of labor by virtue of the commodity's essence and not any qualitative viewpoint.
        \item human labor is \emph{homogeneous} with respect to the commodity.
        \end{itemize}
      \end{remark}

      \begin{definition}
        \label{def:labpow}
        \emph{Labor-power} is a reservior or collective abstraction of human labor, which exists in the organism of every ordinary individual.
      \end{definition}
      \begin{remark}
        We note that:
        \begin{itemize}
        \item This definition seems out of the blue, but since we are not much versed in biology, the words taken as face-value should serve aptly for these notes.
        \item Conversely, organism of every ordinary individual can be viewed as container of the average labor-power\footnote{Within the context of Society.}.
          \item In a society, the organism at certain post for instance air chief marshal etc play a specified \emph{role}, on the other hand, is merely a man/woman (?), a shabby role\footnote{As put forward by Karl.} as a mere human labor.
        \end{itemize}
      \end{remark}
      
      \begin{definition}
        \label{def:values}
        \emph{Values} are mere congelation (semi-solidification) of homogeneous human labor and labor-power expended irrespective of \emph{mode} of its expenditure.
      \end{definition}

      \begin{remark}
        \label{rem:values}
        We note that
        \begin{itemize}
        \item Karl fished out this definition of Values by some ``Social Substance crystal''.  I need to understand what it exactly is.
        \item At this point he kind of stresses that Values are abstracted from use values.  It simply means, Values and use values are to be treated seperately as abstractions.
          \item It is a commodity form, or a property of commodity?
        \end{itemize}
      \end{remark}

      \begin{lemma}
        \label{lemma:valinexval}
        Values of commodities are expressed/manifest themselves in context of exchange values when commodities are exchanged.
      \end{lemma}

      \begin{proof}
        From definition \ref{def:exchval} (of exchange value) we demand that they be transitive.  As the consequence, when commodities (with of course different exchange values) are exchanged, they would still have a common substance, which we can \emph{naturally} identify as Value. 
      \end{proof}

      \begin{remark}
        We make the folowing statements
        \begin{itemize}
        \item The magnitude of value can be measured by the homogeneous human labor spent on the commodity, which, in turn, is measured by its duration.
        \item The duration can be ``denominated'' by weeks, days, and hours.
          \item Finally, we think and ponder on how to find physical representation of true homogeneous human labor.\footnote{Clearly a procrastinating worker completing the work in indefinite amount of time doesn't enhance the value of commodity.  If anything, it diminishes the value.}  
          \end{itemize} 
        \end{remark}

        \begin{definition}
          The total \emph{labor-power} of a society is the \emph{sum total} of the values of \emph{all} the commodities produced by the same society.  This is the homogenoeus mass of uniform human labor-power.
        \end{definition}

        \begin{remark}
          We note that
          \begin{itemize}
          \item Labor-power is \emph{thought to be} composed of innumerable individual \emph{units}\footnote{\textcolor{red}{The detailed analysis on \emph{unit} should be revealed in the later part of these notes.}}.
            \item Futhermore we assume (in the context of commodity production) that these units are indistinguisahble from each other.  Each representative of \emph{some} average colletive property (of society).
            \item Units instrument the gauging of social standards.  That is, what are the regular working timings and so on.
          \end{itemize}
        \end{remark}
        
        Above discussion begs and then demands the right context for homogeneous human labor representation.  One individual may or maynot be enough, but from experience and Karl's arguments, it \emph{seems}, the right context arises with the \emph{collective contribution} of entities existing in a society.

        Let me explain and elaborate Karl's arguments to generate greater mass appeal!  The substance that generates the value (of commodity) is the uniform expenditure of a unit(?) uniform labor-power in form of homogeneous human labor.  Then given the context of commodity production in a society, the value of a commodity is dictated by the expenditure of human labor-power within the time period\footnote{The period is the one that is \emph{socially necessary} given the socially normal conditions of production with the average degree of skill and intensity prevalant at that time.} resulting from the social standards.  Therefore an individual spending less or more time than that socially gauged magnitude is not really representing the homogenous human labor\footnote{Either he is God or highly irrational person!}.

        \paragraph{Case Study: Spinning Jenny}
        The introduction of power looms in England, around 1765 (\href{https://en.wikipedia.org/wiki/James_Hargreaves}{James Hargreaves}), led to the reduced work for churning yarn into cloth.  Thus, the socially gauged magnitude of time for the commodity production, naturally(?) halved (or some fraction).  On the other hand, the handloom weavers, produced the same commodity in same time as before.  That equates to half of the produce, by social standards, given the same time, thus half the human labor (remember it is homogeneous in all possible ways) expended on handloom materials and cosequently, now, the handloom commodity value is halved.

        \begin{remark}
          We note that
          \begin{itemize}
          \item Since social standards now dictate the time, conditions, and so on for commodity production, we have a context for various possible \emph{social dynamics} to appear and chalk out those standards, which in turn drive the society in \emph{some} direction, in very non-direct sense.
            \item It is easy to see that one of the dynamics can be competition which can and will spark in this context, which can be both blessing and curse\footnote{I take no responsibility for describing competition social dynamics here, because my knowledge is quite incomplete and personally I feel it is highly stupid, in the current form, but existing notion.}.
          \end{itemize}
        \end{remark}

        \begin{lemma}
          As values, all commodities are only definite masses of congealed labor time.
        \end{lemma}
        \begin{proof}
          Easy to visualize from above text and discussions.
        \end{proof}

        Unproved lemma
        \begin{lemma}
          \label{lem:comclass}
          \textcolor{red}{In the context of working standards dictated by social standards, each commodity is to be considered as an average sample of its class.}
        \end{lemma}
        \begin{proof}
          To be chalked out elsewhere, or here elsewhen!
        \end{proof}
        \begin{remark}
          \label{rem:comclass}
          We note that
          \begin{itemize}
          \item I don't know what \emph{class} means here.
          \item This lemma deals with social \emph{viewpoint} of the commodity.
            \item In this context, commodities become indistiguisable \emph{if} produced in the \emph{same amount}\footnote{Karl is not serious about the quality of labor time though, here.  It seems that time is outside to his theory and taken to be granted no matter where it comes from.  I'd say that be dictated by the nature of society.} of time.
          \end{itemize}
        \end{remark}

\section{Origins of Definitions and their contexts}
I won't recommend this section for crash course'ers.  Only if you have enough patience and care, you may proceed.
\subsection{Two-fold nature of labor}
The human labor in abstract can be expressed in
\begin{itemize}
\item use value
\item value
\end{itemize}
of a commodity with very different characteristics.  Karl claims that it is due to the fundamental \emph{two-fold} nature of labor itself when studied in the context of commodities.

Consider two commodities, as use values, as following
\begin{itemize}
\item a coat, $\Phi$
\item 10 yards of linen, $\Xi$
\end{itemize}
and let there be a context\footnote{Later we will identify this context by exchange value or value relation.}, in which
\begin{equation}
  \label{eq:exvalphxi}
  \Phi = 2v \text{ and } \Xi = v,
\end{equation}
where, $v$ is some positive number in $\mathbb{R}$ accompanied by some units\footnote{Since we are already familiar with the notion of exchange values (see \ref{def:exchval}), it is not very difficult to imagine use values having same units.}.

\begin{definition}
  \label{def:usefullabor}
  The labor which is responsible for making a thing (\ref{def:thing}), a use value, is called \emph{useful labor}.
\end{definition}

\begin{remark}
  \label{rem:exval}
  We note that:
  \begin{itemize}
  \item Useful labor maifests itself in the utility of the thing (which makes it a use value).
    \item Useful labor is productive activity of definite kind and exercised with a definite aim.
  \item In the above example, $\Phi$ and $\Xi$ are different from each other, qualitatively, in a generalized context.  This implies that the labor responsible for them, correspondingly, should be of different forms i.e tailoring and weaving.
  \item There is no point of exchange between same (indistinguisahble) use values\footnote{Other than checking/recognizing the nature of social standards in different areas and so on.}, seen from qualitative viewpoint.
  \item It is natural to associate a bijection (?) with different use values and different kinds of useful labor.
  \item This demands a calssification of useful labor according to the order, genus, species, and variety to which they belong in the social division of labor.
    \item\label{it:divoflabor}  In community of commodity production, the qualitative difference of useful labor, existing in variety of forms and carried on independently by \emph{individual} producers with accountability, developes into a system primarily supported by a \emph{structure} called division of labor.
  \end{itemize}
\end{remark}


``\emph{\textcolor{blue}{The division of labor is necessary condition for the commodity production.  The commodity production is not necessary condition for divison of labor.}}''\footnote{This makes me realize that division of labor is a structure rather than a mere condition.  See remark \ref{it:divoflabor}.}

For latter, different factors/systems such as biology may be at play\footnote{Think in evolutionary terms and that might lead to something interesting.}.

``\emph{\textcolor{blue}{The relation between physical product (or use value or commodity?) and labor is invariant of the nature of labor, in the sense, that it might be special, by being independent branch of the social division of labor}}.''

This needs more explaination: the relation is not affected by the circumstance in which the physical product is \emph{new}, out of the context of mode of production generated by the division of labor.  In such case, the nature of human race itself imposes some constraints/incentives to generate a relevant context for the ecapsulation of same product production.

``\emph{\textcolor{blue}{That (physical product) which is not a spontaneous produce of Nature, shall owe their existence to some special productive activity, with a definite aim, that appropriates the naturally occuring materials to particular human wants.}}''

Now, as long as labor's useful labor characterstic is manifest, it is a necessary and Nature imposed condition (constraint, independent of all forms of society) for the existence of human race, without which there is no material exchange between human and Nature, and, consequently, no life.

Let me focus on the things, $\Phi$ and $\Xi$, from value (see definition \ref{def:values}) perspective, which is a measure of the homogeneous human labor and gauged or estimated by $v$\footnote{\textcolor{red}{Strictly speaking, the \emph{value} is not equal to $v$}.  It is an abstract notion, not having any dimension}.  In this quantitative context, an exchange value (see \ref{def:exchval}) is natural (given the context of equations \ref{eq:exvalphxi})
\begin{equation}
  \label{eq:exchange}
  \Phi = 2\cdot\Xi = 2v
\end{equation}

It can be easily seen, as value, that they both objectively represent the same homogeneous labor expressed in some form.  We refer the reader to equation \ref{eq:exchangemean} and relevant context for precise meaning.

It is not too difficult to imagine a state of society in which a single man can take on different useful labors (tailoring and weaving for instance here), alternatively.  So now, the two forms of labor are mere modifications of the same homogeneous labor (invoked in the individual) and not a special function of different individuals/persons(?).  \textcolor{red}{I need to find a conceret defition of special activity.}

\emph{In a Capitalist society, the given portion of human labor is in accordance with the variable demand.}

``\emph{\textcolor{blue}{The homogeneous human labor power \ref{def:labpow} must attain a threshold pitch before it can be expended in multiplicity of modes (representing variety of useful labor).}}''

For experimental and educational purposes, let me consider a self-made notion of \emph{work-field}.  We are already familiar with the homogeneous labor in abstract and the notion of labor-power.  Thus it seems natural to define the collection of containers, of labor-power, along with some mechanism (of establishing a relation\footnote{In simple words, it is ``the human labor is expenditure of labor-power''.}) of extracting the human labor, the work-field.
\begin{remark}
  We note that:
  \begin{itemize}
  \item The simple labor power, averaged over the containers in a society, naturally varies in character in different countries and at different times\footnote{I am impressed, Karl!  You are now considering the nature of time in this critique.}.
  \item Skilled labor is simple labor \emph{intensified}, in some useful manner.  One way is
    \begin{equation}
      \text{given quantity of skill } \geq \text{quantity of simple labor}
    \end{equation}
    I will be making it precise in future.
  \end{itemize}
\end{remark}

Now Karl demonstrates a seemingly general scenario (which checks with other video lectures), \emph{that increase in the use value of commodity leads to the decline of its value} (page 33).  It has to do with the two-fold nature of the labor.  I will need to put put this on rigid mathematical formulation.  I wonder if this has some connection to the Spinning-Jenny demonstration.

\subsection{The form of Value and Exchange-Value}
From definition \ref{def:values}, it is clear that values have \emph{reality} in \emph{pure} social context.  Furthermore, we witnessed the expression of commodity values in the context of exchange values \ref{lemma:valinexval}, which is reinforcement of the same idea (that they (values) being substance of social context).  This is the first instance of the inclusion of term ``\emph{\textcolor{red}{money form}}'', synonymous to the varied \emph{bodily} forms of the commodity use value,  as the expression of value.  Let us go back to exchange value equation \ref{eq:exchange} and provide a meaning to it
\begin{equation}
  \label{eq:exchangemean}
  \underbrace{\Phi}_{\text{relative form}} = \underbrace{2\cdot\Xi}_{\text{equivalent form}},
\end{equation}
where $\Phi$ and $\Xi$ represent a coat and 10 yards of linen respectively (the relation of use values).  In this context, besides being pure mathematical formulation\footnote{We imply this to be a usual mathematical equation obeying the assumptions of substitution and equivalence (written \href{https://en.wikipedia.org/wiki/Equality_(mathematics)\#Basic_properties}{here}).}, we say that ``\emph{coat expresses its value in linen}'' by the virtue of exchange value.  Thus the value can be easily expressed in both \emph{relative} and \emph{equivalent} forms which are inseparable and mutually exclusive, within the same context.

\begin{remark}
  We note that
  \begin{itemize}
\item the equations like
\begin{equation}
  \label{eq:tuto}
  \Phi = \Phi,
\end{equation}
are expressing only the indistiguishability of use values, which, from pure mathematics, is obvious, but highly useful and fit for the context of social and working standards touched in remarks \ref{rem:exval} (also mentioned in \ref{lem:comclass} from different context or POV).
\item The summoning of value abstraction nessiciates the notion of exchange value because use value by itself is abstracted from value.
\item Mathematics again comes to our rescue in reversing the relation to
  \begin{equation}
    2\cdot\Xi = \Phi,
  \end{equation}
  and restoring the harmony amongst commodities allowing them to switch roles in whatever fashion the context demands.
  \item From the meaning of equation \ref{eq:exchangemean} it is clear that we can't have same form of value expression on both sides.
  \item We declare that value has two anti-podal forms of expression, namely, relative and equivalent.
\end{itemize}
\end{remark}

In what follows, we would like to understand the the manifestation of value (in the context of commodity exchange) in a quantitative way, thus unearthing the role of society in a very explicit way, enough to introduce the mathematical structure and understand the economical dynamics in a rigid fashion.  In our journey, we will encounter the (in)famous money form and try mold it from pure mathematical standing.

\subsubsection{Nature of \emph{relative} form}
First we revisit the definition of exchange value (or value relation) (see \ref{def:exchval}).  A high-schooler, passionate about Physics, can and will tell you that, in order to compare two quantities, we must first devise the relevant units, on which, comparison can be made\footnote{Length can never be compared with speed or energy (except when you are in High-Energy regime, but that is besides the point).}.  Therefore, a relation like \ref{eq:exchangemean} demands the basic units, if we want the same Physics rigor here.

It is clear that this is a game of contexts, meaning, we are bringing two commodities in a single context of exchange values, using or leveraging the use values (a qualitative view-point).  So we will \emph{fix} the physical dimesion\footnote{If you are unaware of dimensional analysis, please consult \href{https://en.wikipedia.org/wiki/Dimensional_analysis}{this}.} of use values to be $[\omega]$\footnote{I am so tempted to use `uv' initials here, but umm $\ldots$.}.  \textcolor{red}{I have reasons to think that this fixing should hold in general, outside the context of exchange value relations.  I leave it on the time for scrutiny}.

Ok, it turns out I was neglecting the use value as representation of sum of physical properties, which, by the virtue, demands seperate dimensions for different commodities, not brought or existing in the same context.  Hence we will denote the physical dimension of use value to be
\begin{equation}
  \label{eq:usevaldim}
  \left[\Omega_{\text{Commodity Name}}\right]
\end{equation}


The naming convention of units, of dimension $\Omega_{\text{Commodity Name}}$, shall be
\begin{equation}
  \label{eq:nameconvunit}
  \text{`definite number'}\text{`initials of physical measure'}\footnote{Could be something like km, s, or kg}
\end{equation}
Thus for the thing `linen', as it stands alone as commodity, we have a sequence\footnote{Refer the definition \href{https://en.wikipedia.org/wiki/Sequence}{here}.} $\mathbb{S}_{\text{linen}}$ of use values, like so
\begin{equation}
  \label{eq:metricsysuseval}
  \mathbb{S}_{\text{linen}} =  \{\ldots,6\text{yd}, 4\text{yd}, \text{yd}, \ldots\},
\end{equation}
where, `yd' stands for yards, with each element having dimension $\Omega_{\text{linen}}$ and the following usual conversion
\begin{equation}
  \label{eq:usevalconv}
  (6\text{yd})_{\text{linen}} = 6\cdot (\text{yd})_{\text{linen}}.
\end{equation}

If the use value is gauged without any physical measure, then we write `np' for non-physical\footnote{Implying real but not having physical quantity like meters, seconds, ampere etc.}.

Now equipped with this arsnel, we look at equation \ref{eq:exchangemean} again, which translates as follows
\begin{align}
  \label{eq:usevalexeq1}
  \Phi &= 2\cdot \Xi,\\
  \label{eq:usevalexeq2}
  \underbrace{(\text{np})_{\text{coat}}^{\omega}}_{\text{relative form}} &= \underbrace{2\cdot (10\text{yd})_{\text{linen}}^{\omega}}_{\text{equivalent form}}.
\end{align}

Here we have explicitly shown the fixing of use value dimension such that the appropriate comparison is legal\footnote{From pure mathematical POV.}.  In Karl's language, this is equivalent of ``coat=linen'' basis.

\begin{remark}
  We note that, in the context of the relation above
  \begin{itemize}
  \item Linen is the mode of existence of value, implying value is now ``in-action'', as long as dimension of use value is $\omega$.  Linen is the body of value now.  I'd like to think value is \emph{instantiated} on the linen (as equivalent form) by the coat (as relative form), or linen is ``possessed'' by coat, in a useful way.  Hence the \emph{two-fold} nature of  relative form is clear!
  \item On the other hand the value of coat acquires an \emph{independent} expression, in this context, as something comparable and, thus, exchangeable with something different\footnote{Different not in physical sense, in which they really are and we are not interested in that here, but, rather, in the sense of substance and/or different specialized labor.  \textcolor{red}{I am not sure if this is a necessary condition though}.} of equal value.
  \item The value of coat stands forth in its character by the reason of its relation to the other.  Thus we have successfully captured the abstract notion of value in a physical form\footnote{Form following Laws of Physics, strictly.}.
  \item This relation also brings forth the nature of labor common to both varieties of specialized labor, responsible for physical existence and value creation of both things, a human labor in the abstract.
    \item It is the human labor power expended, the human labor in the abstract, which generates the value (as per the definition \ref{def:values}).
    \item In order to define value as congealed state of human labor, it must have objective existence, \emph{materially} different from the coat and yet something common to all the commodities.  This is somehow related to the fact that in equation \ref{eq:usevalexeq1}, $\Xi$, as equivalent, represents something more in the bodily form\footnote{Same as physical form, just sounds more extravagant!  I will be using the two terminologies interchangeably, in these notes.}, a dressed thing, when compared with as standalone commodity.  I may need to find mathematical basis for the expression of this!
    \item $\Xi$ is a body of value, depository of human labor (weaving, a shape of human labor power, expended)
    \item When two things are brought in context, say $A$ and $B$, A can't fulfil its role unless $B$ ``sees'' some bodily form of the same role in $A$.  I know how it works, just want to understand its relevance here (look what I did!).  Would be useful to understand in the reflexive scenario!  When $A=B$.
      \item In Karl's fancy description, commodities talk and communicate when brought/written in value relation.  Furthermore, they do that with the human languages including Hebrew, German, and hopefully Sanskrit!  It would be dope to associate Linguistics with Economics, in the form of component, though!
  \end{itemize}
\end{remark}

\begin{definition}
  \label{def:relval}
  In the exchange value or value relation equation (such as \ref{eq:exchangemean}), the value appearing in the relative form is the \emph{relative value} (a real number in $\mathbb{R}$\footnote{An element of real \emph{field}, to be pedantic, which can be refered \href{https://en.wikipedia.org/wiki/Field_(mathematics)}{here}}).
\end{definition}

\begin{remark}
  We note that
  \begin{itemize}
  \item It is important to distinguish and, thus, clarify the meaning of relative value.  The value (see \ref{def:values}), of the commodity in relative form (see \ref{eq:usevalexeq2}), is \emph{expressed} in the use value (see \ref{def:usevalue}) of the commodity in the equivalent from (again see \ref{eq:usevalexeq2}), and we say that the ``value has taken the form of \emph{relative value}''\footnote{It is very important to understand theorem \ref{th:relvalandval} before making any prejuidices about it.}.
  \end{itemize}
\end{remark}

Now relative value is something numerical and we denote it by $v_{\text{coat}}^{r}$.  We consider equation \ref{eq:usevalexeq2} again with a scope of ``kinematics''\footnote{In Physics, kinematics is the analysis of evolving system only in terms of predefined degrees of freedom, with relevant time, without the regard for internal/external influences and backreactions.  The analysis done after solving Equations Of Motion, so to speak!}
\begin{equation}
  \label{eq:valreltestvar}
  (\text{np})_{\text{coat}}^{\omega} = 2\cdot F(\ldots)\cdot (10\text{yd})_{\text{linen}}^{\omega},
\end{equation}
where, $F(\ldots)$ is called \emph{form factor} whose arguments\footnote{We note that this same factor can be quite useful in expressing the dependence of exchange value, of a commodity, on time and place, as merrily noted in \ref{def:exchval}.  I would like to even go further by stating that ``\textcolor{blue}{$F(\ldots)$ is supported on pesudo-Riemannian manifold $(\mathcal{M}, g)$''}.  Here $g$ is the solution to Equations of Motion resulting from the Einstein-Hilbert action.} we will try to determine by variational methods, such that, previous arguments hold, in the context of exchange values.


\begin{itemize}
\item First we demand the following explicit dependence
  \begin{equation}
  v_{\text{coat}}^{r} = F(t_{\text{coat}}^l, \ldots) = a\cdot t_{\text{coat}}^l, 
\end{equation}
where $t_{\text{coat}}^l$ is the amount of time, the human labor (expended human labor power) was congealed for the production of coat.
It is easy to check that once $t_{\text{coat}}^l$ doubles, it has direct effect of same factor in the equation \ref{eq:valreltestvar}.

Conclusion:``\emph{\textcolor{blue}{The value of commodity rises $\Rightarrow$ relative value $v^r$ may\footnote{We will be making weak statements for now, because complete form of $v^r$ is unknown, given the factor $a$.} increase.}}''

\item Next, we demand the following explicit dependence of $a$ like so
  \begin{equation}
    \label{eq:adep}
    a = \lambda\cdot\frac{1}{t_{\text{linen}}^l},
  \end{equation}
  resulting in the equation
  \begin{equation}
    \label{eq:relvalform}
     v_{\text{coat}}^{r} = \lambda \cdot \frac{t_{\text{coat}}^l}{t_{\text{linen}}^l}.
   \end{equation}
   
   Conclusion:``\emph{\textcolor{blue}{The value of commodity remains fixed $\Rightarrow$ relative value $v^r$ may decrease.}}''
 \end{itemize}
 I am doing nothing new, just framing Karl's words in pure mathematics.  The new part is leaving the $\lambda$ as is, because, in future, if we find more factors responsible for value change of a commodity (favorite endeavor for mathematics friends), it would be easy enough to incorporate in this encapsulation of form factor!

 \paragraph{Properties of $v_{\text{Commodity Name}}^{r}$}
 This subsubsub-section is partly less scrutnized, because, I need to do more thinking.

 \begin{theorem}
   \label{th:relvalandval}
 Change in $v^{r}$ (given the mathematical form \ref{eq:relvalform}) is neither necessary nor sufficient condition for reflecting change in the value of the commodity.\footnote{The symbol for relative value, of generic commodity, is written by ommiting the sub-text.}
 \end{theorem}

 From Karl's arguments and common sense, we see if the labor time is increasing simultaneously for both the commodites, in exchange relation \ref{eq:valreltestvar}, in \emph{same direction} and with \emph{same proportions}, then one can easily imagine values of both the commodities increasing but the $v_{\text{coat}}^{r}$, defined in \ref{eq:relvalform}, remains fixed.

 \begin{proof}
   Consider the following (mathematical) variation of relative value (given the expression in \ref{eq:relvalform})
   \begin{align}
     \label{eq:variationdemo}
      \delta v_{\text{coat}}^{r} &= \lambda\frac{t_{\text{linen}}^l\delta{t_{\text{coat}}^l} - t_{\text{coat}}^l\delta{t_{\text{linen}}^l}}{\left(t_{\text{linen}}^l\right)^2}
   \end{align}
   Now as per Karl's arguments if the relevant conditions are met (same direction and same proportion)
   \begin{align}
     \delta{t_{\text{coat}}^l} &= \gamma \delta{t_{\text{linen}}^l},\\
     \gamma &= \frac{t_{\text{coat}}^l}{t_{\text{linen}}^l},
   \end{align}
   it is easy to see that $\delta v_{\text{coat}}^{r}$ vanishes in-spite of $ \delta{t_{\text{coat}}^l}, \delta{t_{\text{linen}}^l} \neq 0$, which represent not-trivial, but simple, human labor time changing the values of commodities written in value relation.

   So we have seen that $\delta v^r = 0$ doesn't necessarily imply constant absolute value of the commodity.  Earlier we saw that even $\delta v^r \neq 0$ doesn't always imply changing absolute value of commodity.
 \end{proof}

 Moving on, since we have discovered the conditions of degenracy in $v_{\text{coat}}^r$, the natural question is: how to revive the relative value?
 Solution is simple, yet involves introduction of more complexity!  That is, bringing third commodity, say pants, in the context of exchange values, whose value is constant.  \textcolor{red}{I think that $v^r$ needs better naming, in terms of all commodities involved!}


\subsubsection{Nature of \emph{equivalent} form}
Let me rewrite the exchange value equation \ref{eq:valreltestvar} to paint the picture.
\begin{equation}
  \label{eq:exvalrep}
   (\text{np})_{\text{coat}}^{\omega} = 2\cdot F(\ldots)\cdot (10\text{yd})_{\text{linen}}^{\omega}.
 \end{equation}
 \begin{remark}
   Since we have fixed the dimension of both the use values as $[\omega]$, we note that, in this context of exchange value (of coat?)
   \begin{itemize}
   \item Coat expresses its ``quality of having a value'' (relative value $v^r$), in the use value of other commodity (equivalent form), by the fact that coat is still in the bodily form (completely explainable as per the Laws of Physics) and nothing more is required from the relation.
   \item The same bodily form is now the value form (an equivalent form), we say, only when some \emph{other} (product of different kind of labor?) commodity enters the value relation in relative form.
   \item The fact that linen has value is clear from the relation which allows the direct exchange with coat.
   \item Linen, as a seperate commodity, has its value based on the labor time\footnote{And whatnot!  I need to think more about it.}.  But when written in equivalent form, value can't acquire any quantitative form (unlike $v^r$ for relative form)\footnote{What Karl seems to be suggesting is that, when the labor time in equivalent commodity is increased, in the context of this relation, it just diminishes the relative value of relative commodity without expressing itself in equivalent form.  It checks out with the variational kinematics of equation \ref{eq:valreltestvar}.}.  The value magic is gone for the equivalent commodity and, on the other hand, it gained the magic of being in physics realm\footnote{I'd like to think of it as collection of some vibrational mode of (bosonic?) strings, from String Theory.}, as a definite quantity.
   \item \textcolor{blue}{One clue, about bringing two commodities in exchange value context, is the deciding criteria of the ratio (or initial value of form factor $F(\ldots)$) which is the magnitude (whatever it means) of absolute value.}.
   \item We can also see the form factor $F(\ldots)$ as the ``equivalent use value embodiement'' of value (expression in form of $v^r$).
   \item The use value (from which the Value was abstracted\footnote{See Values remark \ref{rem:values}.}) becomes the menifestation of its opposite, the Value.  However keep in mind, use value of equivalent form and Value of relative form of different(?) commodities are being identified.
     \item Now Karl states something I highly disagree with, and this sheds light on the tension between unchecked Economics and mathematical form of Economics\footnote{Compilation of the entire scope in mathematical language, inevitably from first principles.  Axiomatic Economics, but that would be ambitious.}.  ``No commodity can stand in the (relative) relation of equivalent to itself, and thus turn its own bodily shape into the expression of its own value, every commodity is \emph{compelled} to choose some other commodity \ldots''.  I can't convince myself, especially when I perform a literal \emph{Gedankenexperiment} of standing in front of (physical) mirror and see my reflection.  Here I am the relative form and reflection is equivalent form, which is essentially my body, in some sense, being the mode of my-being (the Value) expression.  If you want mathematical foundation of doubt, refer equation \ref{eq:usevalexeq2}, which by mathematics, should hold \emph{reflexive} property (clearly visible in \ref{eq:tuto}).
   \end{itemize}
   \end{remark}


\bibliographystyle{JHEP.bst}
\bibliography{list.bib}
\end{document}